\documentclass[a4paper,12pt]{article}
\usepackage{polski}
\usepackage[utf8]{inputenc}
\usepackage{graphicx}
\usepackage{bm}
\usepackage{amsmath}
\usepackage{indentfirst}
\usepackage{float}
\usepackage{url}
\usepackage{latexsym}

\begin{document}

\part*{Pralka}
\noindent Aleksandra Prodziewicz, Zuzanna Poznańska\\
\tableofcontents

\section{Definicja projektu}

\subsection{Cel projektu}

Głównym celem projektu jest zebranie dokumentacji analityczno-projektowej do budowy i implementacji urządzenia piorącego. Pozwoli to na ułatwienie codziennej pracy, jaką jest robienie prania, zoptymalizowanie zasobów oraz czasu.


\subsection{Zakres projektu}

Zakres projektu obejmuje stworzenie kompletnego rozwiązania projektowego dla urządzenia piorącego.

\noindent W skład rozwiązania wchodzą:\\
$\bullet$ analiza systemu\\
$\bullet$ stworzenie specyfikacji wymagań funkcjonalnych i niefunkcjonalnych \\
$\bullet$ opracowanie i oszacowanie nakładów, wymaganych do stworzenia takiego systemu\\


\section{Opis ogólny}

\subsection{Perspektywa produktu}

Po analizie rynku i dostępnych urządzeń, stwierdza się, że nie istnieje w chwili obecnej urządzenie, które pozwoli na zautomatyzowane zrobienie prania. Stworzenie własnego modelu pralki okazało się najlepszym rozwiązaniem. Pozwoli to spotkać w 100$\%$ wymagania użytkowników i pogodzić to z ich preferencjami.\\
Spodziewa się, że stworzenie własnego modelu przyniesie następujące korzyści:\\
$\bullet$ standaryzacja procesów i obsługi\\
$\bullet$ poprawa jakości\\
$\bullet$ uporządkowanie dokumentacji



\subsection{Użytkownicy systemu}



\section{Specyfikacja wymagań}

\subsection{Wymagania funkcjonalne}

\subsection{Wymagania niefunkcjonalne}



\section{Funkcjonalności systemu}

\subsection{Przypadki użycia}

\subsection{Klasy}


\section*{Bibliografia}
\begin{enumerate}
\item \url{http://www.math.uni.wroc.pl/~mpal/academic/2001/analiza/tydz1415.pdf?fbclid=IwAR3A5xrCyMpSRHmQVs-2dzBhSjTDo2FWb1pVLw1BEbzc-k-wnEBg6DPuYTQ}
\item \url{http://maciej.grzesiak.pracownik.put.poznan.pl/W-ANA-calki-krzyw.pdf?fbclid=IwAR33TuBbn9y85isaZopwe88q9DTXrFrgFQ4caOq4ZOA1HZ0CFdm1Tbf6wBA}
\item \url{http://www.pg.gda.pl/snm/pracownicy/anita.tlalka/analiza_wektorowa_ck.pdf}
\end{enumerate}
\end{document}